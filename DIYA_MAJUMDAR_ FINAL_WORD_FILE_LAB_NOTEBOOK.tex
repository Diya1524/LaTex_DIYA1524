% Generated by GrindEQ Word-to-LaTeX 
\documentclass{article} % use \documentstyle for old LaTeX compilers

\usepackage[utf8]{inputenc} % 'cp1252'-Western, 'cp1251'-Cyrillic, etc.
\usepackage[english]{babel} % 'french', 'german', 'spanish', 'danish', etc.
\usepackage{amsmath}
\usepackage{amssymb}
\usepackage{txfonts}
\usepackage{mathdots}
\usepackage[classicReIm]{kpfonts}
\usepackage{graphicx}

% You can include more LaTeX packages here 


\begin{document}

%\selectlanguage{english} % remove comment delimiter ('%') and select language if required


\noindent \textbf{Maulana Abul Kalam Azad University of Technology, WB}

\noindent \textbf{}

\noindent \includegraphics*[width=1.25in, height=1.25in]{image1}\textbf{}

\noindent \textbf{}

\noindent \textbf{}

\noindent 
\subsection{Software Tools and Technology}

\noindent \textbf{Lab Notebook:-}

\noindent \textbf{}

\noindent \textbf{Group Assignment}

\noindent \textbf{}

\noindent \textbf{}

\noindent \textbf{Repository Link: https://github.com/Diya1524/LaTex\_GROUP\_ASSIGNMENT\_DIYA1524/new/main}

\noindent 

\noindent 

\noindent \textbf{Group Members}

\noindent 

\noindent \textbf{}

\begin{enumerate}
\item \textbf{ DIYA MAJUMDARRoll No: 30085323010   }
\end{enumerate}

\noindent \textbf{Department: BSC IN IT CYBERSECURITY}

\noindent \textbf{}

\noindent \textbf{}

\noindent \textbf{Instructor:  }Ayan Ghosh

\noindent \textbf{\textit{Date: September 13, 2024}}

\noindent \textbf{\underbar{Index}}

\noindent \textbf{}

\noindent \textbf{}

\begin{tabular}{|p{0.9in}|p{3.3in}|} \hline 
\textbf{Serial No.} & \textbf{Questions} \\ \hline 
1 & Introduction to GitHub and GitHub Desktop version installation \\ \hline 
2 & Building a C program for a calculator in the local repository,committing,and publishing it as a public repository \\ \hline 
3 & Converting a submit button to Chin Tapak Dum Dum \\ \hline 
4 & Build a cv in latex \\ \hline 
5 & Branching and Merging \\ \hline 
\end{tabular}

\textbf{}

\noindent \textbf{}

\noindent \textbf{}

\noindent \textbf{}

\noindent \textbf{}

\noindent \textbf{}

\noindent \textbf{}

\noindent \textbf{}

\noindent \textbf{}

\noindent \textbf{}

\noindent \textbf{}

\noindent \textbf{}

\noindent \textbf{}

\noindent \textbf{}

\noindent \textbf{}

\noindent \textbf{}

\noindent \textbf{}

\noindent \textbf{}

\noindent \textbf{}

\noindent \textbf{}

\noindent \textbf{}

\noindent \textbf{}

\noindent \textbf{}

\noindent \textbf{}

\noindent \textbf{}

\noindent \textbf{}

\noindent \textbf{}

\noindent \textbf{}

\noindent \textbf{}

\noindent \textbf{}

\noindent \textbf{}

\noindent \textbf{}

\noindent \textbf{}

\noindent \textbf{}

\noindent \textbf{}

\noindent \textbf{}

\noindent \textbf{}

\noindent \textbf{}

\noindent \textbf{}

\noindent \textbf{}

\noindent \textbf{}
\[1\] 

\section{Lab Notebook Entries}

\noindent 
\paragraph{Entry by DIYA MAJUMDAR}

\noindent \textit{Date: [September 21, 2024]}

\noindent \textit{}

\noindent \textit{}

\noindent \includegraphics*[width=3.20in, height=1.80in]{image2}\textit{}

\noindent \textit{}

\noindent \textit{}

\noindent \textit{}

\noindent \textbf{\underbar{-:GitHub:-}}

\noindent GitHub is a web-based platform that allows developers to host, share, and collaborate on software projects. It provides a version control system powered by Git, enabling

\noindent teams to track changes, manage code repositories, and work together seamlessly across different locations. GitHub supports collaborative development through features like pull requests, issues, and project boards, making it essential for open-source projects and professional software development. Additionally, it integrates with various development tools, enhancing productivity and streamlining the software development lifecycle.

\noindent 

\noindent 

\noindent 

\noindent \textbf{\underbar{-:Installation:-}}

Installing GitHub Desktop is a straightforward process that enhances your workflow by providing a user-friendly interface for managing repositories. To begin, download the

\noindent installer from the [official GitHub Desktop website](https://desktop.github.com/) for your operating system---Windows or macOS. After downloading, run the installer and follow the on-screen instructions to complete the setup. Once installed, launch the application and sign in with your GitHub credentials, or create a new account if needed.

\noindent GitHub Desktop simplifies the process of cloning repositories, making commits, and managing branches, making it an invaluable tool for developers of all skill levels. For Linux users, alternative methods like using Wine or other Git clients are available.

\noindent 

\noindent 

\noindent 

\noindent 

\noindent 

\noindent 

\noindent 
\[2\] 

\paragraph{Entry by DIYA MAJUMDAR}

\noindent \textit{Date: [September 20, 2024]}

\noindent \textit{}

\noindent 
\paragraph{0.1  Building a C program for a calculator in the local repository,committing,and publishing it as a public}

\noindent \textbf{repository}

\noindent \textbf{1 Introduction}

\noindent This document outlines the process of creating a basic calculator in C, committing it to a local Git repository, and pushing it to a public repository on GitHub.

\noindent 

\noindent 
\subsection{2 Step 1: Writing the C Program}

\noindent Below is the source code for a simple calculator written in C:

\noindent \includegraphics*[width=3.74in, height=5.84in]{image3}

\noindent 

\noindent 

\noindent Figure 1: Screenshot of the source code.

\noindent 

\noindent 
\[3\] 

\subsection{3 Step 2: Create a New Repository on GitHub}

1. Go to your GitHub profile: GitHub -- DIYA1524. 2. Click on New to create a new repository. 3. Name the repository C-Calculator and ensure it is set to Public. 4. Do

\noindent not initialize the repository with a README, .gitignore, or license, as these are not needed for the local repository you have created.

\noindent 

\noindent 
\subsection{4 Step 3:  Push Local Repository to GitHub}

\noindent Copy the remote URL of the new GitHub repository. It should look like this: [caption=Example remote URL] https://github.com/Diya1524/calculator.c

\noindent Add the remote repository to your local Git: [language=bash, caption=Adding a remote repository] git remote add origin https://github.com/Diya1524/calculator.c

\noindent Push your local commits to the GitHub repository: [language=bash, caption=Pushing to GitHub] git push -u origin master

\noindent 

\noindent 
\subsection{5 Step 4:  Verify on GitHub}

\noindent Visit your GitHub profile to ensure that the code is now available in your public repository:  https://github.com/Diya1524/calculator.c

\noindent 

\noindent 

\noindent 

\noindent 

\noindent 

\noindent 

\noindent 

\noindent 

\noindent 

\noindent 

\noindent 

\noindent 

\noindent 

\noindent 

\noindent 

\noindent 

\noindent 

\noindent 

\noindent 

\noindent 

\noindent 

\noindent 

\noindent 

\noindent 

\noindent 

\noindent 

\noindent 

\noindent 
\[4\] 

\paragraph{Entry by DIYA MAJUMDAR}

\noindent \textit{Date: [September 20, 2024]}

\noindent \textit{}

\noindent 
\subsection{-:Converting a submit button to ''chin-tapak- dumdum'':-}

\noindent \textbf{}

\noindent Steps to Fix the Button and Create a Pull Request

\begin{enumerate}
\item  Clone the Repository: - You've already cloned the repository using GitHub Desktop, so you should have the project on your local machine.
\end{enumerate}

\noindent 

\noindent 

\begin{enumerate}
\item  Open the Project: - Open the project in your preferred IDE as per the instructions in the readme.md.
\end{enumerate}

\noindent 

\noindent 

\begin{enumerate}
\item  Locate the Button Code: - Search for the code where the submit button is defined.
\end{enumerate}

\noindent This is typically found in the frontend code of the application. Depending on the technology stack used (e.g., HTML/CSS, React, Angular, etc.), it could be in a file like index.html, App.js, ButtonComponent.js, or a similar file.

\noindent 

\noindent 

\begin{enumerate}
\item  Rename the Button: - You renamed the button to ''Chin-Tapak-DumDum''. If the button's size became disproportionate, it's likely due to styling issues.
\end{enumerate}

\noindent 

\noindent 

\begin{enumerate}
\item  Test the Application: - Run the application again to ensure that the button appears correctly and that there are no additional issues.
\end{enumerate}

\noindent 

\noindent 

\begin{enumerate}
\item   Commit Your Changes: - Once the button looks good, commit your changes. Use descriptive commit messages, for example: Fixed button styling after renaming.
\end{enumerate}

\noindent bash git add . git commit -m ''Fixed button styling after renaming to 'Chin Tapak Dum Dum'''

\noindent 

\noindent 

\begin{enumerate}
\item  Push Your Changes: - Push your changes to your forked repository on GitHub. bash git push origin main
\end{enumerate}

\noindent (Replace main with the correct branch name if it's different.)

\noindent 

\noindent 

\begin{enumerate}
\item  Create a Pull Request: - Go to the original repository on GitHub (the one you cloned from). - You should see an option to create a pull request from your forked repository.
\end{enumerate}

\noindent Follow the instructions to create a pull request with a title and description of what you have done.

\noindent Make sure to mention in the pull request that you have fixed the button styling after renaming it.

\noindent 
\[5\] 

\paragraph{Entry by DIYA MAJUMDAR}

\noindent \textit{Date: [September 20, 2024]}

\noindent \textit{}

\noindent 
\paragraph{5.1  LaTeX is a typesetting system widely used for producing professional documents. It is especially useful for creating}

\noindent \textbf{\includegraphics*[width=4.48in, height=6.31in]{image4}structured, well-formatted documents like academic papers, reports, and CVs. LaTeX allows you to focus on the content and logical structure of your document while it handles the layout.}

\noindent \textbf{}

\noindent \textbf{}

\noindent \textbf{}

\noindent \textbf{}

\noindent Figure 1: Screenshot of CV which is done in latex format.

\noindent 

\noindent 

\noindent 

\noindent 
\[6\] 
\textbf{Steps to Build a CV in LaTeX Here is a step-by-step guide to building a CV using LaTeX:}

\noindent \textbf{}

\noindent \textbf{Step 1: Install LaTeX Install LaTeX distribution on your system: Windows: Install MiKTeX. macOS: Install MacTeX. Linux: Install TeX Live using the package manager (e.g., sudo apt install texlive-full. You can also use an online LaTeX editor like Overleaf}

\noindent \textbf{}

\noindent \textbf{}

\noindent \textbf{}

\noindent \textbf{}

\noindent \textbf{}

\noindent \textbf{}

\noindent \textbf{}

\noindent \textbf{}

\noindent \textbf{}

\noindent \textbf{}

\noindent \textbf{}

\noindent \textbf{Step 2: Compile the LaTeX Code Use a LaTeX editor like Overleaf or compile the .tex file on your local machine with commands like:}

\noindent \textbf{}

\noindent \textbf{}

\noindent 

\noindent 

\noindent 

\noindent 

\noindent 

\noindent 

\noindent 

\noindent 

\noindent 

\noindent 

\noindent \textbf{Step 3: Customize and Add More Sections Add Experience: Include jobs, internships, and volunteering. Add Education: Include degrees, courses, or certifications. Add Projects: List important technical projects, with links if they are on GitHub. Skills Achievements: Highlight technical and}

\noindent \textbf{non-technical skills. After compiling, a PDF file is generated which can be shared or printed as your CV.}

\noindent 

\noindent 

\noindent 

\noindent 

\noindent 

\noindent 

\noindent 

\noindent 
\[7\] 


\noindent \textbf{}

\noindent \textbf{}

\noindent \textbf{}

\noindent \textbf{}

\noindent \textbf{}

\noindent 
\paragraph{Entry by DIYA MAJUMDAR}

\noindent \textit{Date: [September 20, 2024]}

\noindent \textit{}

\noindent 
\section{Git Assignment 3 : Branching and Merging}

\noindent \textbf{Objective: }Demonstrate proficiency in Git branching, merging, and conflict resolution.

\noindent 

\noindent \includegraphics*[width=5.12in, height=2.88in]{image5}

\noindent 

\noindent 

\noindent Figure 5: Screenshot of the GitHub repository showing the commit history.

\noindent 

\noindent 

\noindent 

\noindent 

\noindent 

\noindent 

\noindent 

\noindent 

\noindent 

\noindent 

\noindent 

\noindent 

\noindent 

\noindent 

\noindent 

\noindent 

\noindent 

\noindent 
\[8\] 
\includegraphics*[width=4.98in, height=2.80in]{image6}

\noindent 

\noindent 

\noindent \includegraphics*[width=5.12in, height=2.88in]{image7}Figure 6: Screenshot of the GitHub repository showing the branching.

\noindent 

\noindent Figure 7: Repository after deleting branches 'feature-1' and 'feature-2'.

\noindent 

\noindent 

\noindent 

\noindent 

\noindent 

\noindent 

\noindent 

\noindent 

\noindent 
\[9\] 


\noindent 
\section{Write-Up: Experience with Git Branching and Merging}

\noindent \textbf{}

\noindent This assignment focused on the use of Git branching and merging to manage features in a collaborative environment. After creating separate branches for feature-1 and feature-2, each was developed independently. When merging feature-1 into the main

\noindent branch, there were no conflicts. However, the merge of feature-2 caused a conflict in the shared.txt file. Conflict resolution was done manually, ensuring that the changes from both branches were retained. The practical experience highlighted the importance of clear commit messages and Git's branching capabilities for parallel development and conflict management.

\noindent 

\noindent 

\noindent 

\noindent 

\noindent 

\noindent 

\noindent 

\noindent 

\noindent 

\noindent 

\noindent 

\noindent 

\noindent 

\noindent 

\noindent 

\noindent 

\noindent 

\noindent 

\noindent 

\noindent 

\noindent 

\noindent 

\noindent 

\noindent 

\noindent 

\noindent 

\noindent 

\noindent 

\noindent 

\noindent 

\noindent 

\noindent 
\[10\] 



\end{document}

